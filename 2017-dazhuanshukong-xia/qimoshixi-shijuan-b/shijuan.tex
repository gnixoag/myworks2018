\documentclass[12pt,twocolumn,landscape,UTF8,twoside]{ctexart}
\input{data/shijuan_ys}

\setboolean{print}{true}
%\setboolean{print}{false} %是否打印答案


\author{高星}

\begin{document}
	\noindent	
	
	\begin{spacing}{1.2}
		\begin{center}
			\zihao{3} \heiti 
			湖南潇湘技师学院~~湖南九嶷职业技术学院
			
			\uline{\makebox[35mm][c]{2017-- 2018}}学年 \hspace{0.5cm} 第\,\uline{\makebox[10mm][c]{二}}期末考试试卷	
			
			\zihao{-4} \songti \vspace{5mm}
			\begin{tabu} to 0.45\textwidth {X[1.3,l]X[1,l]X[0.8,l]}
				
				课程名称:\zihao{5}  \uline{\makebox[35mm][c]{数铣实习}}&
				课程编号:\uline{\makebox[20mm][c]{~}} &
				\uline{~~~~~~~~~B~~~~~~~~~~}卷\\ 
				
				考试班级:\zihao{5} \uline{\makebox[35mm][c]{2017级大专模具班}}& 
				考试方式:\uline{\makebox[20mm][c]{实训}}& 	
				拟卷人:\uline{\makebox[18mm][c]{高星} } \\ 
				
				拟卷日期:\uline{\makebox[35mm][c]{2018.12.15}}& 
				审核人:\uline{\makebox[25mm][c]{}} & 	
				审核日期:\uline{\makebox[15mm][c]{}} \\ 
			\end{tabu}  
			
			
			\zihao{-4} \songti \vspace{5mm}
%			\begin{tabu} to 0.45\textwidth {|X[2,c]|X[1,c]|X[1,c]|X[1,c]|X[1,c]
%					|X[1,c]|X[1,c]|X[2,c]|}
%				\hline 
%				题\hfill 号&一&二&三&四&五&六& 总\hfill 分\\ 
%				\hline 
%				得\hfill 分&  &  &  &  &  &  &   \\ 
%				\hline 
%				评\hfill 卷\hfill 人&  &  &  &  &  &    &  \\ 
%				\hline 
%				复\hfill 查\hfill 人&  &  &  &  &  &    &  \\ 
%				\hline 
%			\end{tabu} 
		\end{center}
	\end{spacing}
	
	\vspace{-8pt}\zihao{5} 

\begin{spacing}{1.3}
	\begin{enumerate} [1、]
		\item[\heiti 一、] {\heiti 实习考试内容}

 从下面两个图中随机选取一个,在数控铣床完成零件的加工,毛胚为\diameter 110$\times$40.5的尼龙棒。
 考试时间为100分钟。
\begin{figure}[h]
	\centering
	\includegraphics[width=0.5\linewidth,trim=220 10 250 10,clip]{image/3.jpg}
	\caption{零件1}
	\label{fig:3}
\end{figure}

\begin{figure}[h]
	\centering
	\includegraphics[width=0.5\linewidth,trim=220 10 250 10,clip]{image/4.jpg}
		\caption{零件2}
	\label{fig:3}
\end{figure}

%\item[\heiti 二、] {\heiti 要求}
%\item 合理选择加工刀具与切削用量。
%\item 合理的计算及选择刀补。
%\item 要使用自动换刀及长度补偿指令。
%\item 以自己的姓名保存项目到最后一个盘,并上传到服务器。
%\item O1号程序用来连接各个程序,如下。
%\begin{lstlisting}
%O1	(主程序,安排加工顺序)
%M98P2	(铣上表面)
%M98P3	(钻中心孔)
%M98P4	(钻孔)
%M98P5	(铰孔)
%M98P6	(粗铣外形)
%M98P7	(粗铣槽)
%M00	(精加工前暂停)
%M98P8	(精铣外形)
%M98P9	(精铣槽)
%M30
%\end{lstlisting}
%
%\vspace{2cm}


\newpage

\item[\heiti 二、] {\heiti 评分标准}

\zihao{4} 
\begin{tabu} to 0.45\textwidth {|X[0.8,c]|X[1.5,c]|X[2,c]|X[1,c]|X[1,c]
	|X[1,c]|X[1,c]|}
	\hline 
	序号&内容&检测内容&配分&得分&备注\\ \hline
	1&平面&程序及尺寸&15&&\\ \hline
	2&外形&程序及尺寸&20&&\\ \hline
	3&槽&程序及尺寸&30&&\\ \hline
	4&孔&程序及尺寸&20&&\\ \hline
	5&文明安全&&15&&\\ \hline		
\end{tabu} 

	\end{enumerate} 
	\end{spacing}
\end{document}