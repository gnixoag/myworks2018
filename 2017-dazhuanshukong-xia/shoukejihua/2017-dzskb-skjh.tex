\documentclass{ctexart}

\usepackage{ifthen}

\input{data/en_preamble}

%%%% 正文部分
\begin{document}
%%%%%%%%----------------------------------------------------
%%%% 开始首页
\xn{2018--2019} %学年
\xq{1} %学期
\xb{机电工程系} %系部
\zy{数控技术} %专业
\bj{2017级大专数控班
%	\par 2017级大专数控班 
%	\par 2017级大专数控班 
} %班级
\kc{《数控编程与实习》} %课程
\skzs{17} %上课周数
\zxs{4[2](4)} %周学时
\jk{×} %讲课
\sy{×} %实验
\lljk{64} %理论讲课
\sx{\zihao{5} [30]\par(64)} %实训
\sxlljk{×} %实习理论讲课
\scsx{×} %生产实习
\kh{2[4]} %考核
\jd{2(4)} %机动
\hj{\zihao{5}68\par[34]\par(68)} %合计
\ywcks{无} %已完成课时
\ylks{224}  %余留课时
\khfs{理论考试/实习考查}  %考核方式
\jcmc{数控机床编程与操作-数控铣床/加工中心分册~张梦欣}  %教材名称

\jhsy %生成计划首页
%%%% 结束首页
%%%%%%%%----------------------------------------------------

%%%%%%%%----------------------------------------------------
%%%% 计划说明
\begin{center}
\zihao{2} \heiti 学期授课计划说明
\end{center}
\zihao{-4} \setlength{\parindent}{2em} \setlength{\baselineskip}{22pt}

\textbf{一、教学目的与要求:}

本学期主要学习数控编程的基础知识,
要求学生能熟练的掌握好数控编程的基础和一些简化指令的基本方法和思路去编程和解决实际问题,
充分把自己 的能力及智慧通过编程展示出来。
为后续的数控学习作好准备。

\textbf{二、用教材、参考书}

1、使用教材: 《数控机床编程与操作(数控铣床 加工中心分册)》张梦欣主编

2、参考书:《加工中心编程与操作》~~  科学出版社 ~~ 刘加孝主编

\hspace{5em}《加工中心操作工》~~ 中国劳动社会保障出版社 ~~ 杨伟群主编

\hspace{5em}《数控铣床加工实训--中级模块》 ~~劳动版 ~~史建军主编

\textbf{三、教学措施}

1、采用多媒体、仿真、讨论等教学方法,本课程结合蓝墨云班课进行信息化的教学。

2、作业:理论课每周布置一道编程题,仿真每周做习题集上的题目,实习除了完成课题外,还要每个课题写一个实习报告。

3、学生评价采用自评、小组评价、教师评价三结合。

4、成绩平定,采用百分制,平时占70\%,包括出勤,作业,课堂答问等,期末闭卷占30\%。

\textbf{四、增删内容}

本计划无增删内容。

\textbf{五、本课程与其他课程的关系}

本课程是专业课,其他课程是基础,为本课服务。先要学习好《数控加工工艺》、《普铣》、《机械制图》、《机械加工原理》、《专业数学》等课程。在这些课程的基础上再来学习本课程就容易多了,希望同学们多复习这些课程。

\textbf{六、课程计划周数:}

授课时间为2-19周(第1周老生生报道,第5周国庆放假1周),上课周数17周,周课时10节。

\onecolumn \setlength{\parindent}{0em}
%%%% 计划说明
%%%%%%%%----------------------------------------------------

\newcounter{lilunN} %理论编号
\setcounter{lilunN}{0} %重置编号为0

\newcommand{\llh}{ \stepcounter{lilunN} 理论\arabic{lilunN}、} %新建理论编号命令
\newcommand{\syt}{ 自绘示意图\arabic{lilunN} }%新建自绘示意图命令
\newcommand{\xt}{ 习题\arabic{lilunN} }%新建习题命令

\newcounter{zhou} \setcounter{zhou}{0}%周 
\newcounter{ci} [zhou] \setcounter{ci}{0}%次 
\newcommand{\zc}{\ifthenelse{ \not \isodd{\value{ci}}}{\stepcounter{zhou}\setcounter{ci}{0}}{} \stepcounter{ci} \arabic{zhou}/\arabic{ci}} %新建周次命令(默认一周2次),显示:周/次
\newcommand{\zcd}{\stepcounter{zhou} \arabic{zhou}} %新建单独周命令,一周一次课

%%%%%%%%----------------------------------------------------
%%%% 开始教学计划表
\begin{jxjhb}
\zcd&新生报到、教师备课		& & & & & 09.03 09.09 & \\[6ex] \hline

\zc& \llh  数控编程概术 & 掌握数控编程的基本知识&\syt &\xt & 2节 & 09.10-09.16 &  \\[6ex] \hline
\zc& \llh 程序的基本结构 &掌握程序结构及了解基本指令 &\syt &\xt & 2节 &  & \\[6ex] \hline

\zc& \llh 基本指令及写程序思路 &掌握基本指令及编程基本思路 &\syt &\xt  &2节 & 09.17 09.23 & \\[6ex] \hline
\zc& \llh 基本指令(一)	&掌握G0/G1指令及了解G2G3指令 &\syt &\xt  & 2节& & \\[6ex] \hline

\zc& \llh 基本指令(二)	& 掌握G2/G3指令的两种用法&\syt &\xt  &2节 & 09.24 09.30 & \\[6ex] \hline
\zc& \llh 基本指令(三) & 掌握G2/G3指令的编程&\syt &\xt & 2节& & \\[6ex] \hline

\zcd& 国庆放假		 & & & & & 10.01 10.07& \\[6ex] \hline

\zc& \llh 倒角与拐圆角 &掌握倒角与拐圆角使用 &\syt  &\xt  &2节 &10.08 10.14 & \\[6ex] \hline
\zc& \llh 刀具半径补偿 &掌握用半径补偿指令编程 & \syt & \xt & 2节&   & \\[6ex] \hline

\zc& \llh 刀具半径补偿的应用& 掌握用刀补进行加工精度的控制&自绘示意图8 &习题8 &2节& 10.15 10.21& \\[6ex] \hline
\end{jxjhb}

\begin{jxjhb}

\zc& \llh 子程序的XY向分层 &掌握用子程序进行XY向分层& \syt &\xt  &2节 & & \\[6ex] \hline

\zc& \llh 变量编程概述 &掌握变量编程的基本知识 & \syt &\xt  & 2节& 10.22 10.28 & \\[6ex] \hline
\zc& \llh 宏程序Z向分层 & 掌握宏程序Z向分层的编程& \syt &\xt  &2节 & & \\[6ex] \hline

\zc& \llh 技能抽查实例(一) & 掌握技能抽查实例外形的编程& \syt &\xt  &2节 &10.29 11.04 & \\[6ex] \hline
\zc& \llh 技能抽查实例(二)& 掌握技能抽查实例外形的编程& \syt &\xt  &2节 & & \\[6ex] \hline


\zc& \llh 圆形槽加工& 掌握圆形槽的编程&\syt  &\xt  &2节&11.05 11.11 & \\[6ex] \hline
\zc& \llh 矩形型腔加工& 掌握矩形型腔的编程&\syt  &\xt  &2节&   & \\[6ex] \hline

\zc& \llh 凸轮槽加工(正负刀补)& 掌握正负刀补的应用 &\syt  &\xt  &2节 &  11.12 11.18& \\[6ex] \hline
\zc& \llh 孔加工概述 & 掌握孔加工工艺及编程思路& \syt & \xt & 2节& & \\[6ex] \hline
		
\zc& \llh Fanuc 上的孔加工循环(一) 	& 掌握Fanuc上孔加工指令&\syt  &\xt  &2节 & 11.19 11.25 & \\[6ex] \hline
\zc& \llh Fanuc 上的孔加工循环(二) &掌握Fanuc孔的编程& \syt &\xt &2节 &  & \\[6ex] \hline

\end{jxjhb}

\begin{jxjhb}	
\zc& \llh 长度补偿概述 & 掌握长度补偿指令的使用& \syt &\xt  &2节 &11.26 12.02 & \\[6ex] \hline	
\zc& \llh 长度补偿的应用 &掌握用长度补偿来编程 &\syt  &\xt  & 2节& & \\[6ex] \hline	
	
\zc& \llh 椭圆加工 &掌握用宏程序来编写椭圆程序 &\syt  &\xt  & 2节&12.03 12.09 & \\[6ex] \hline	
\zc& \llh 技能抽查实例(三) & 掌握技能抽实例的编程& \syt &\xt  &2节 & & \\[6ex] \hline

\zc& \llh 技能抽查实例(四) & 掌握技能抽实例的编程& \syt &\xt  &2节 & 12.10 12.16& \\[6ex] \hline
\zc& \llh 技能抽查实例(五) & 掌握技能抽实例的编程& \syt &\xt  &2节 &  & \\[6ex] \hline

\zc& \llh 技能抽查实例(六) & 掌握技能抽实例的编程& \syt &\xt  &2节 & 12.17 12.23& \\[6ex] \hline
\zc& \llh 技能抽查实例(七) & 掌握技能抽实例的编程& \syt &\xt  &2节 & & \\[6ex] \hline	

\zc& \llh 西门子系统的编程 &掌握西门子系统指令的区别& \syt &\xt  &2节 & 12.24 12.30& \\[6ex] \hline
\zc& \llh 局部坐标系 & 掌握局部坐标系的使用& \syt & \xt & 2节& & \\[6ex] \hline		
	
\zc& \llh 坐标系旋转 &掌握坐标系旋转指令的使用& \syt &\xt  &2节 &12.31 01.06& \\[6ex] \hline

		
\end{jxjhb}
\begin{jxjhb}
	
\zc& \llh 期末复习 &复习本学期所学知识 &\syt &\xt  & 2节& & \\[6ex] \hline
\zcd& 期末考试、阅卷、成绩登录 & & && &01.07 01.11& \\[6ex] \hline	
	
	&  & & & & & & \\[6ex] \hline
	&  & & & & & & \\[6ex] \hline
	&  & & & & & & \\[6ex] \hline
	&  & & & & & & \\[6ex] \hline
	&  & & & & & & \\[6ex] \hline
	&  & & & & & & \\[6ex] \hline
	&  & & & & & & \\[6ex] \hline
	&  & & & & & & \\[6ex] \hline	
	
\end{jxjhb}
\shqz %审核签字

\begin{jxjhb}
	1&老生报到、教师备课		& & & & & 09.03 09.09 & \\[6ex] \hline
	
	2& 实习1、安全操作及机床面板认识 &认识机床及掌握基本操作&数控机床及\par 相关工具 &实习报告1 & [2](4)& 09.10 09.16& \\[6ex] \hline
	
	3& 实习2、程序手工录入、编辑及刀路模拟 &掌握程序的快速录入与编辑 &数控机床及\par 相关工具 & 实习报告2& [2](4)& 09.17 09.23& \\[6ex] \hline
	
	4& 实习3、工件找正、装夹与对刀、调速 &掌握工件的对刀方法(取中法) &数控机床及\par 相关工具 &实习报告3 &  [2](4) & 09.24 09.30& \\[6ex] \hline

	5& 国庆放假 & & & & & 10.01 10.07& \\[6ex] \hline

	6& 实习4、面铣及手动铣削 &会铣平面与简单零件加工 &数控机床及\par 相关工具 &实习报告4 &  [2](4) & 10.08 10.14& \\[6ex] \hline
	
	
	7-10& 实习5、外轮廓加工 &掌握子程序的编程应用与精度控制\par 掌握增加刀路对零件去残料&数控机床及\par 相关工具 &实习报告5 &  [8](16)& 10.15 11.11& \\[6ex] \hline
	
	11-14& 实习6、型腔加工 &掌握挖槽加工工艺与精度控制\par 掌握岛屿类零件的编程与加工	&数控机床及\par 相关工具 &实习报告6 &  [8](16)& 11.12 12.09& \\[6ex] \hline
	
	15-18& 实习7、技能抽查综合加工&掌握孔的加工\par 	掌握技能抽查零件的加工	&数控机床及\par 相关工具 &实习报告7 &  [8](16)& 12.10 01.06 & \\[6ex] \hline

	19&  期末考试、阅卷、成绩登录 & & && &01.07 10.13 & \\[6ex] \hline	
	
\end{jxjhb}

%
%\begin{jxjhb}
%
%
%	
%	&  & & & & & & \\[6ex] \hline
%	&  & & & & & & \\[6ex] \hline
%	&  & & & & & & \\[6ex] \hline
%	&  & & & & & & \\[6ex] \hline
%	&  & & & & & & \\[6ex] \hline
%	&  & & & & & & \\[6ex] \hline
%	&  & & & & & & \\[6ex] \hline
%	&  & & & & & & \\[6ex] \hline
%	
%\end{jxjhb}

\shqz %审核签字


%%%% 教学计划表结束
%%%%%%%%----------------------------------------------------

\end{document}
%%%% 正文部分结束
%%%%%%%%----------------------------------------------------