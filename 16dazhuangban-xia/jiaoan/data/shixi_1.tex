\jxhj{%教学后记
}
\skrq{%授课日期
2018.9.10~|~2018.9.17~|~2018.9.24~|~2018.10.8~|~ 1-4节}
\ktmq{%课题名称
UG 二维自动编程}
\jxmb{%教学目标,每行前面要加\item 
\item 掌握 UG 自动编程模块基本使用;
\item 掌握UG 平面铣参数的含义及设置;
\item 掌握UG 平面铣各子项的应用场合及区别;
\item 掌握哑铃的加工工艺与加工。}
\jxzd{%教学重点,每行前面要加\item 
\item UG 平面铣参数的含义及设置;
\item 哑铃的加工工艺与加工。}
\jxnd{%教学难点,每行前面要加\item 
\item UG 平面铣参数的含义及设置;}
\jjff{%教学方法
通过讲述、举例、演示法来说明;}

\makeshouye%制作教案首页

%%%%教学内容
\subsection{实习教学要求}
\begin{compactenum}[1、]
\item 掌握 UG 自动编程模块基本使用;
\item 掌握UG 平面铣参数的含义及设置;
\item 掌握UG 平面铣各子项的应用场合及区别;
\item 掌握哑铃的加工工艺与加工。
\end{compactenum}

\subsection{相关工艺}
\subsubsection{UG数控加工的操作流程}
\subsubsection{UG平面铣各参数含义及设置}
\begin{compactenum}[1、]
\item 部件边界;
\item 毛坯边界;
\item 底平面;
\item 夹具体;
\item 修剪边界;
\item 加工参数;
\item 非切削移动参数;
\item       机床控制。
\end{compactenum}
\subsubsection{平面铣各子类型的使用}

\subsubsection{加工实例}\marginpar{具体后面讲解}


\subsection{实习内容及过程}

\subsubsection{集合、组织实习}
1、清查学生人数

2、文明安全生产讲解

3、实习内容说明
\subsubsection{开机15分钟}
1、由组长记录机床相关问题

2、开机前检查仔细

3、空转几分钟预热
\subsubsection{机床操作及编程}
1、教师演示基本操作

2、组长安排2人员操作机床(1人操作,1个指导)

3、其他人员自选图形编程

4、每人操作时间不得超过2小时

5、教师巡回指导
\subsubsection{操作点评及工件检测}
1、学生操作感想说明及自评

2、教师提问及点评

3、学生对工件自测

4、教师检测及评分
\subsubsection{准备下课}
1、清洁数控机床

2、正常关机

3、集合教师点评

\subsection{练习题及作业}

1、完成蓝墨云班课的作业


\vfill
\subsection{加工准备与加工要求}
\subsubsection{加工准备}
\begin{enumerate}[1、]
\item 设备:数控铣床、加工中心。
\item 材料:45圆钢(150*120*30)。
\item  工具:活动扳手,平行垫铁,百分表,其它常用辅具。
\item  
量具:外径千分尺(0~25、100~125,0.01),深度千分尺(0~25,0.01),R规。
\item  刀具:$\phi$ 8、$\phi$ 12立铣刀、$\phi$ 20机夹刀。
\item  夹具:三爪自定心卡盘、螺杆压板、平口钳。
\end{enumerate}
\subsubsection{课题评分表}

{\noindent
\begin{figure}[!hbtp]
%\centering
\footnotesize
\hspace{-3.4ex}\renewcommand\arraystretch{1.9}
\begin{tabu}to0.45\textwidth{|cc|c|c|c|c|c|c|}
\hline
\multicolumn{2}{|c|}{工件编号}&\multicolumn{2}{c}{}&
\multicolumn{2}{|c}{总得分}&\multicolumn{2}{|c|}{}\\
\hline
\multicolumn{2}{|c|}{项目与配分}&\parbox{2ex}{序号}&技术要求&配分
&评分标准&\parbox{4ex}{检测记录}&得分\\
\hline
\multicolumn{2}{|c|}{\multirow{2}{*}{\parbox{10ex}{文明生产
			(20\%)}}}&1&工作服&8&未穿禁止进车间并全扣&&\\
\cline{3-8}
&&2&工具、量具摆放整齐&8&不整齐有序全扣&&\\ \cline{3-8}
&&3&其它&4&不守纪律全扣&&\\
\hline

\multicolumn{2}{|c|}{\multirow{1}{*}{\parbox{10ex}{安全操作规程
			(20\%)}}}&4&操作安全&20&酌情扣分&&\\[.3cm] \hline

\multicolumn{2}{|c|}{\multirow{1}{*}{\parbox{10ex}{加工中心组成
			(10\%)}}}&5&说出各部分名称&10&出错一处扣2分&&\\[.3cm]  \hline

\multicolumn{2}{|c|}{\multirow{2}{*}{\parbox{10ex}{面板系统界面
			(20\%)}}}&6&操作面板&10&出错一处扣2分&&\\ \cline{3-8}
&&7&系统界面的认识	&10&出错一处扣2分&&\\ \hline

\multicolumn{2}{|c|}{\multirow{4}{*}{\parbox{10ex}{机床操作
			(30\%)}}}&8&手动、手轮、快速&10&出错一处扣2分&&\\ \cline{3-8}
&&9&点的定位&10&出错一处扣2分&&\\ \cline{3-8}		
&&10&机床操作规范&5&出错一处扣2分&&\\ \cline{3-8}		
&&11&工件刀具装夹	&5&出错一处扣2分&&\\ \hline	

\multicolumn{2}{|c|}{\multirow{2}{*}{\parbox{10ex}{安全文明生产
(倒扣分)}
}}&12&安全操作&倒扣&
\multirow{2}{*}{\parbox{14ex}{安全事故停止操作或酌情扣分}}&&\\
\cline{3-5}\cline{7-8}
&&13&机床整理&倒扣&&&\\
\hline
\end{tabu}
\end{figure}}