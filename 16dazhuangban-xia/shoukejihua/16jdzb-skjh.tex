
\documentclass{ctexart}

\input{data/en_preamble}

%%%% 正文部分
\begin{document}
%%%%%%%%----------------------------------------------------
%%%% 开始首页
\xn{2018--2019} %学年
\xq{1} %学期
\xb{机电工程系} %系部
\zy{模具设计与制造} %专业
\bj{16级大专模具班} %班级
\kc{《数控编程》} %课程
\skzs{17} %上课周数
\zxs{(4)} %周学时
\jk{×} %讲课
\sy{×} %实验
\lljk{×} %理论讲课
\sx{ (60)} %实训
\sxlljk{×} %实习理论讲课
\scsx{×} %生产实习
\kh{(4)} %考核
\jd{(4)} %机动
\hj{(68)} %合计
\ywcks{260} %已完成课时
\ylks{0}  %余留课时
\khfs{考查}  %考核方式
\jcmc{数控机床编程与操作-数控铣床/加工中心分册~沈建峰}  %教材名称

\jhsy %生成计划首页
%%%% 结束首页
%%%%%%%%----------------------------------------------------

%%%%%%%%----------------------------------------------------
%%%% 计划说明
\begin{center}
\zihao{2} \heiti 学期授课计划说明
\end{center}
\zihao{-4} \setlength{\parindent}{2em} \setlength{\baselineskip}{22pt}

\textbf{一、教学目的与要求:}

上学期主要学习了自动编程,所用的软件为 MasterCAM X7,同学们掌握的还行。
本学期主要在上个学期的基础上学习Siemens NX软件的自动编程,要求学生能熟练运用各种自动编程方法来解决实际问题,充分把自己的能力及智慧通过编程展示出来。为以后走上工作岗位作好准备。

\textbf{二、用教材、参考书}

1、使用教材: 《数控机床编程与操作(数控铣床 加工中心分册)》 沈建峰

2、参考书:《加工中心编程与操作》  科学出版社  刘加孝   主编

\hspace{5em}《加工中心操作工》 中国劳动社会保障出版社  杨伟群  主编

\hspace{5em}《加工中心考工实训教程》  化学工业出版社   吴明友 主编

\textbf{三、教学措施}

1、采用多媒体、讨论等教学方法、运用蓝墨云班课软件进行信息化管理。

2、作业:实习除了完成加工课题外,还要每个课题写一个实习报告,另外会利用蓝墨云班课软件布置课内外作用。

3、学生评价采用自评、小组评价、教师评价三结合。

4、成绩平定,采用百分制,平时占70\%,包括出勤,作业,课堂答问等。

\textbf{四、增删内容}

本计划无增删内容。

\textbf{五、本课程与其他课程的关系}

本课程是专业课,其他课程是基础,为本课服务。先要学习好《数控加工工艺》、《普铣》、《机械制图》、《机械加工原理》、《专业数学》等课程。在这些课程的基础上再来学习本课程就容易多了,希望同学们多复习这些课程。

\textbf{六、课程计划周数:}

授课时间为2--19周(第1周老生报道,第5周国庆放假,第19周期末理论考试),上课周数17周,周课时4节。

\onecolumn \setlength{\parindent}{0em}
%%%% 计划说明
%%%%%%%%----------------------------------------------------

%%%%%%%%----------------------------------------------------
%%%% 开始教学计划表
\begin{jxjhb}
	
	1& 新生上课,老师报到注册 	& & & & & 09.03 09.09& \\[6ex] \hline
	
	2-6& 课题1、UG二维自动编程加工 &掌握UG 自动编程模块基本使用\par 掌握哑铃片的加工工艺 &数控机床及\par 相关工具 &实习报告1 & (16)&09.10 10.14& \\[6ex] \hline
	
	7-10& 课题2、UG三维自动编程加工 &掌握UG 三维造型\par 掌握UG三维曲面加工 &数控机床及\par 相关工具 & 实习报告2& (16)& 10.15 11.11& \\[6ex] \hline
	
	11-14& 课题3、UG综合加工 & 掌握UG自动编程的部分技巧 \par 能自己用UG进行编程加工  & 数控机床及\par 相关工具 &实习报告3 &  (16) & 11.12 12.09& \\[6ex] \hline
	
	15-18& 课题4、考证强化训练 &  掌握高级工考证题的造型与加工\par 掌握高级工考证题的加工工艺  &数控机床及\par 相关工具 &实习报告4 &  (16)& 12.10 01.06& \\[6ex] \hline

	19& 期末考试、阅卷 & & & & (4)&01.07 01.13 & \\[6ex] \hline
	
	& & & & & & & \\[6ex] \hline
	
	& & & & & & & \\[6ex] \hline
	
	& & & & & & & \\[6ex] \hline
	
	& & & & & & & \\[6ex] \hline
	
\end{jxjhb}

\shqz %审核签字


%%%% 教学计划表结束
%%%%%%%%----------------------------------------------------

\end{document}
%%%% 正文部分结束
%%%%%%%%----------------------------------------------------