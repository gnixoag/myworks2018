
\documentclass{ctexart}

\input{data/en_preamble}

%%%% 正文部分
\begin{document}
%%%%%%%%----------------------------------------------------
%%%% 开始首页
\xn{2017--2018} %学年
\xq{2} %学期
\xb{机电工程系} %系部
\zy{模具设计与制造} %专业
\bj{16级大专模具班} %班级
\kc{《数控编程》} %课程
\skzs{15} %上课周数
\zxs{[2](3)} %周学时
\jk{×} %讲课
\sy{×} %实验
\lljk{×} %理论讲课
\sx{\zihao{5} [30]\par(45)} %实训
\sxlljk{×} %实习理论讲课
\scsx{×} %生产实习
\kh{0} %考核
\jd{0} %机动
\hj{\zihao{5}(30)\par[45]} %合计
\ywcks{180} %已完成课时
\ylks{90}  %余留课时
\khfs{考查}  %考核方式
\jcmc{数控机床编程与操作-数控铣床/加工中心分册~沈建峰}  %教材名称

\jhsy %生成计划首页
%%%% 结束首页
%%%%%%%%----------------------------------------------------

%%%%%%%%----------------------------------------------------
%%%% 计划说明
\begin{center}
\zihao{2} \heiti 学期授课计划说明
\end{center}
\zihao{-4} \setlength{\parindent}{2em} \setlength{\baselineskip}{22pt}

\textbf{一、教学目的与要求:}

本学期主要在上个学期的基础上学习数控编程中的自动编程,要求学生能熟练运用各种自动编程方法来解决实际问题,充分把自己的能力及智慧通过编程展示出来。为以后走上工作岗位作好准备。

\textbf{二、用教材、参考书}

1、使用教材: 《数控机床编程与操作(数控铣床 加工中心分册)》 沈建峰

2、参考书:《加工中心编程与操作》  科学出版社  刘加孝   主编

\hspace{5em}《加工中心操作工》 中国劳动社会保障出版社  杨伟群  主编

\hspace{5em}《加工中心考工实训教程》  化学工业出版社   吴明友 主编

\textbf{三、教学措施}

1、采用多媒体、仿真、讨论等教学方法。

2、作业:理论课每周布置一道编程题,仿真每周做习题集上的题目,实习除了完成课题外,还要每个课题写一个实习报告。

3、学生评价采用自评、小组评价、教师评价三结合。

4、成绩平定,采用百分制,平时占70\%,包括出勤,作业,课堂答问等,期末闭卷占30\%。

\textbf{四、增删内容}

本计划无增删内容。

\textbf{五、本课程与其他课程的关系}

本课程是专业课,其他课程是基础,为本课服务。先要学习好《数控加工工艺》、《普铣》、《机械制图》、《机械加工原理》、《专业数学》等课程。在这些课程的基础上再来学习本课程就容易多了,希望同学们多复习这些课程。

\textbf{六、课程计划周数:}

授课时间为2--16周(第1周新生报道,第17周期末理论考试),上课周数15周,周课时5节。

\onecolumn \setlength{\parindent}{0em}
%%%% 计划说明
%%%%%%%%----------------------------------------------------

%%%%%%%%----------------------------------------------------
%%%% 开始教学计划表
\begin{jxjhb}
	
	1& 学生报到注册 	& & & & & 03.05 03.09& \\[6ex] \hline
	
	2-5& 实习1、二维烟灰缸加工 &掌握MasterCAM基本使用\par 掌握烟灰缸加工的加工工艺 &数控机床及\par 相关工具 &实习报告1 & [8](12)&03.12 04.06& \\[6ex] \hline
	
	6-9& 实习2、三维曲面加工 &掌握MasterCAM三维造型\par 掌握三维曲面加工 &数控机床及\par 相关工具 & 实习报告2& [8](12)& 04.09 05.04& \\[6ex] \hline
	
	10-13& 实习3、考证课题1 &掌握高级工考证题的造型与加工\par 掌握高级工考证题的加工工艺 &数控机床及\par 相关工具 &实习报告3 &  [8](12) & 05.07 06.01& \\[4.5ex] \hline
	
	14-16& 实习4、综合实训 &掌握MasterCAM的运用\par 掌握MasterCAM的部分技巧&数控机床及\par 相关工具 &实习报告4 &  [6](9)& 06.04 06.22& \\[4.5ex] \hline

	17& 期末考试、阅卷 & & & & &06.25 06.29 & \\[6ex] \hline
	
	& & & & & & & \\[6ex] \hline
	
	& & & & & & & \\[6ex] \hline
	
	& & & & & & & \\[6ex] \hline
	
	& & & & & & & \\[6ex] \hline
	
\end{jxjhb}

\shqz %审核签字


%%%% 教学计划表结束
%%%%%%%%----------------------------------------------------

\end{document}
%%%% 正文部分结束
%%%%%%%%----------------------------------------------------